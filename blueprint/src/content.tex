% In this file you should put the actual content of the blueprint.
% It will be used both by the web and the print version.
% It should *not* include the \begin{document}
%
% If you want to split the blueprint content into several files then
% the current file can be a simple sequence of \input. Otherwise It
% can start with a \section or \chapter for instance.

\chapter{Introduction}

.....

\chapter{Pigeonhole Principle}

% \begin{theorem}[Smale 1958]
%   \label{thm:sphere_eversion}
%   \lean{sphere_eversion}
%   \leanok
%   \uses{def:immersion}
%   There is a homotopy of immersions of $𝕊^2$ into $ℝ^3$ from the inclusion map to
%   the antipodal map $a : q ↦ -q$.
% \end{theorem}
  
% \begin{proof}
%   \leanok
%   \uses{thm:open_ample, lem:open_ample_immersion}
%   This obviously follows from what we did so far.
% \end

\begin{theorem}
  \label{thm:pigeonhole_principle_simple} 
  \lean{pigeonhole_simple}
  \leanok
  If $n + 1$ objects are distributed into $n$ boxes, then at least one box contains two or more of the objects.
\end{theorem}

\begin{proof}
  \leanok
  Suppose that no box contains more than one object. Then each box contains at most one object, so the total number of objects is at most $n$. This contradicts the assumption that there are $n + 1$ objects.
\end{proof}

\begin{theorem}
  \label{thm:2.2.1}
  For $n$ and $r$ positive integers with $r \leq n$, 
  \[ P(n,r)= n\times(n- 1) \times \cdots \times (n-r+1).  \]
\end{theorem}


