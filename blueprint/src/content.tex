% In this file you should put the actual content of the blueprint.
% It will be used both by the web and the print version.
% It should *not* include the \begin{document}
%
% If you want to split the blueprint content into several files then
% the current file can be a simple sequence of \input. Otherwise It
% can start with a \section or \chapter for instance.

\chapter{Introduction}

.....

% \chapter{Pigeonhole Principle}

% \begin{theorem}[Smale 1958]
%   \label{thm:sphere_eversion}
%   \lean{sphere_eversion}
%   \leanok
%   \uses{def:immersion}
%   There is a homotopy of immersions of $𝕊^2$ into $ℝ^3$ from the inclusion map to
%   the antipodal map $a : q ↦ -q$.
% \end{theorem}
  
% \begin{proof}
%   \leanok
%   \uses{thm:open_ample, lem:open_ample_immersion}
%   This obviously follows from what we did so far.
% \end

% \begin{theorem}
%   \label{thm:pigeonhole_principle_simple} 
%   \lean{pigeonhole_simple}
%   \leanok
%   If $n + 1$ objects are distributed into $n$ boxes, then at least one box contains two or more of the objects.
% \end{theorem}



% \begin{theorem}
%   \label{thm:2.2.1}
%   For $n$ and $r$ positive integers with $r \leq n$, 
%   \[ P(n,r)= n\times(n- 1) \times \cdots \times (n-r+1).  \]
% \end{theorem}



\chapter{What Is Combinatorics?}

\chapter{Permutations and Combinations}

\begin{theorem}
    \label{thm:2.2.1}
    For $n$ and $r$ positive integers with $r\leq n$,
    \[P(n,r) = n\times(n-1)\times\cdots\times(n-r+1).\]  
\end{theorem} 

\begin{theorem}
    \label{thm:2.2.2}
    The number of circular r-permutations of a set of $n$ elements is given by 
    \[\frac{P(n,r)}{r}=\frac{n!}{r\cdot(n-r)!}.\]
    In particular, the number of circular permutations of $n$ elements is $(n - 1)!$.
\end{theorem}

\begin{theorem}
    \label{thm:2.3.1} 
    For $0\leq r \leq n$,
    \[P(n,r)=r!\binom{n}{r}\]
    Hence, 
    \[\binom{n}{r} = \frac{n!}{r! (n-r)!}.\]
\end{theorem}

\begin{corollary}
    \label{cor:2.3.2}
    For $0 \leq r \leq n$,
    \[\binom{n}{r} = \binom{n}{n-r}.\]
\end{corollary}

\begin{theorem}
    \label{thm:2.3.3}
    (\textbf{Pascal's formula}) For all integers $n$ and $k$ with $1 \leq k \leq n-1$,
    \[\binom{n}{k} = \binom{n-1}{k} + \binom{n-1}{k-1}.\]
\end{theorem}

\begin{theorem}
    \label{thm:2.3.4}
    For $n \geq 0$,
    \[\binom{n}{0} + \binom{n}{1} + \binom{n}{2} + \cdots + \binom{n}{n} = 2 ^ {n},\]
    and the common value equals the number of subsets of an n-element set.
\end{theorem}

\begin{theorem}
    \label{thm:2.4.1}
    Let $S$ be a multiset with objects of $k$ different types, where each object has an infinite repetition number. Then the number of r-permutations of $S$ is $k^{r}$.
\end{theorem}

\begin{theorem}
    \label{thm:2.4.2}
    Let $S$ be a multiset with objects of $k$ different types with finite repetition numbers $n_1, n_2, ... , n_k$, respectively. Let the size of $S$ be $n = n_1 + n_2 + ... + n_k$. Then the number of permutations of $S$ equals
    \[\frac{n!}{n_1!n_2!\dots n_k!}.\]
\end{theorem}

\begin{theorem}
    \label{thm:2.4.3}
    Let $n$ be a positive integer and let $n_1, n_2, ...  ,n_k$ be positive integers with $n = n_1 + n_2 + ... + n_k$. The number of ways to partition a set of $n$ objects into $k$ labeled boxes in which Box 1 contains $n_1$ objects, Box 2 contains $n_2$ objects, ..., Box k contains $n_k$ objects equals
    \[\frac{n!}{n_1!n_2!\dots n_k!}.\]
    If the boxes are not labeled, and $n_1 = n_2 = ... = n_k$, then the number of partitions equals
    \[\frac{n!}{k!n_1!n_2!\dots n_k!}.\]
\end{theorem}

\begin{theorem}
    \label{thm:2.4.4}
    There are $n$ rooks of $k$ colors with $n_1$ rooks of the first color, $n_2$ rooks of the second color, . . . , and $n_k$ rooks of the kth color. The number of ways to arrange these rooks on an n-by-n board so that no rook can attack another 
    \[n!\frac{n!}{n_1!n_2!\cdots n_k!}=\frac{(n!)^2}{n_1!n_2!\cdots n_k!}.\]
\end{theorem}

\begin{theorem}
    \label{thm:2.5.1}
    Let $S$ be a multiset with objects of $k$ types, each with an infinite repetition number. Then the number of r-combinations of $S$ equals
    \[\binom{r+k-1}{r}=\binom{r+k-1}{k-1}\]
\end{theorem}

\chapter{The Pigeonhole Principle}

\begin{theorem}
    \label{thm:3.1.1}
    \lean{thm3_1_1}
    \leanok
   If $n + 1$ objects are distributed into $n$ boxes, then at least one box 
contains two or more of the objects.
\end{theorem} 

\begin{proof}
  \leanok
  Suppose that no box contains more than one object. Then each box contains at most one object, so the total number of objects is at most $n$. This contradicts the assumption that there are $n + 1$ objects.
\end{proof}

\begin{theorem}
    \label{thm:3.2.1}
   Let $q_1, q_2, . .. ,q_n$ be positive integers. If
   \[q_1 + q_2 + \cdots + q_n - n + 1\]
objects are distributed into $n$ boxes, then either the first box contains at least $q_1$ objects, or the second box contains at least $q_2$ objects, $\dots$, or the $n$th box contains at least $q_n$ objects.
\end{theorem} 

\begin{corollary}
    \label{cor:3.2.2}
Let $n$ and $r$ be positive integers. If $n(r-1) + 1$
objects are distributed into $n$ boxes, then at least one of the boxes contains $r$ or more of the objects.
\end{corollary}

\begin{theorem}
    \label{thm:3.3.1}
If $m \geq 2$ and $n \geq 2$ are integers, then there is a positive integer $p$ such that
\[K_p \rightarrow K_m, K_n.\]
\end{theorem} 

\chapter{Generating Permutations and Combinations}

\begin{theorem}
    \label{thm:4.2.1}
Let $b_1, b_2, \dots, b_n$ be a sequence of integers satisfying
\[
0 \leq b_1 \leq n-1, \quad 0 \leq b_2 \leq n-2,\quad \dots \quad 0 \leq b_{n-1} \leq 1, \quad b_n = 0.
\]
Then there exists a unique permutation of $\{1, 2, \dots, n\}$ whose inversion sequence is $b_1, b_2, \dots, b_n$.
\end{theorem} 

\begin{theorem}
  \label{thm:4.3.1}
  The preceding algorithm for generating the $n$-tuples of 0s and 1s 
  produces the reflected Gray code of order $n$ for each positive integer $n$.
\end{theorem} 

\begin{theorem}
  \label{thm:4.4.1}
  Let $a_1, a_2 \cdots a_r$ be an $r$-subset of $\{1, 2, \dots, n\}$. 
  The first $r$-subset in the lexicographic ordering is $1 2 \cdots r$. The last $r$-subset 
  in the lexicographic ordering is $(n-r+1)(n-r+2) \cdots n$. Assume that 
  $a_1a_2 \cdots a_r \neq (n-r+1)(n-r+2) \cdots n$. Let $k$ be the largest integer such that 
  $a_k < n$ and $a_k + 1$ is different from each of $a_{k+1}, \dots, a_r$. Then the $r$-subset 
  that is the immediate successor of $a_1a_2 \cdots a_r$ in the lexicographic ordering is
  \[
  a_1 \cdots a_{k-1}(a_k + 1)(a_k + 2) \cdots (a_k + r - k + 1).
  \]
\end{theorem} 


\begin{theorem}
    \label{thm:4.4.2}
The $r$-subset $a_1a_2 \cdots a_r$ of $\{1, 2, \dots, n\}$ occurs in place number
\[
\binom{n}{r} - \binom{n - a_1}{r} - \binom{n - a_2}{r-1} - \cdots - \binom{n - a_{r-1}}{2} - \binom{n - a_r}{1}
\]
in the lexicographic order of the $r$-subsets of $\{1, 2, \dots, n\}$.
\end{theorem} 

\begin{theorem}
    \label{thm:4.5.1}
Let $X$ be a finite set with $n$ elements. Then there is a one-to-one correspondence between the total orders on $X$ and the permutations of $X$. In particular, the number of different total orders on $X$ is $n!$.
\end{theorem} 

\begin{theorem}
    \label{thm:4.5.2}
Let $(X, \leq)$ be a finite partially ordered set. Then there is a linear extension of $(X, \leq)$.
\end{theorem} 

\begin{theorem}
    \label{thm:4.5.3}
Let $\sim$ be an equivalence relation on a set $X$. Then the distinct equivalence classes partition $X$ into nonempty parts. Conversely, given any partition of $X$ into nonempty parts, there is an equivalence relation on $X$ whose equivalence classes are the parts of the partition.
\end{theorem} 

\chapter{The Binomial Coefficients}

\begin{theorem}
    \label{thm:5.2.1}
Let \( n \) be a positive integer. Then, for all \( x \) and \( y \),
\[
(x+y)^{n}=x^{n}+\binom{n}{1}x^{n-1}y+\binom{n}{2}x^{n-2}y^{2}+\cdots+\binom{n}{n-1}x^{1}y^{n-1}+y^{n}.
\]
In summation notation,
\[
(x+y)^n=\sum_{k=0}^n\binom{n}{k} x^{n-k} y^k.
\]
\end{theorem}

\begin{theorem}
    \label{thm:5.2.2}
Let \( n \) be a positive integer. Then, for all \( x \),
\[
(1+x)^n = \sum_{k=0}^n \binom{n}{k} x^k = \sum_{k=0}^n \binom{n}{n-k} x^k.
\]
\end{theorem}

\begin{theorem}
    \label{thm:5.3.1}
Let \( n \) be a positive integer. The sequence of binomial coefficients
\[
\binom{n}{0}, \binom{n}{1}, \binom{n}{2}, \ldots, \binom{n}{n}
\]
is a unimodal sequence. More precisely, if \( n \) is even,
\begin{align*}
\binom{n}{0} &< \binom{n}{1} < \cdots < \binom{n}{n/2}, \\
\binom{n}{n/2} &> \cdots > \binom{n}{n-1} > \binom{n}{n},
\end{align*}
and if \( n \) is odd,
\begin{align*}
\binom{n}{0} &< \binom{n}{1} < \cdots < \binom{n}{(n-1)/2} = \binom{n}{(n+1)/2}, \\
&\binom{n}{(n+1)/2} > \cdots > \binom{n}{n-1} > \binom{n}{n}.
\end{align*}
\end{theorem}

\begin{corollary}
    \label{cor:5.3.2}
For \( n \) a positive integer, the largest of the binomial coefficients
\[
\binom{n}{0}, \binom{n}{1}, \binom{n}{2}, \ldots, \binom{n}{n}
\]
is
\[
\binom{n}{\lfloor n/2 \rfloor} = \binom{n}{\lceil n/2 \rceil}.
\]
\end{corollary}

\begin{theorem}
    \label{thm:5.3.3}
Let \( S \) be a set of \( n \) elements. Then an antichain on \( S \) contains at most \( \binom{n}{\lfloor \frac{n}{2} \rfloor} \) sets.
\end{theorem}

\begin{theorem}
    \label{thm:5.4.1}
Let \( n \) be a positive integer. For all \( x_1, x_2, \ldots, x_t \),
$
(x_1 + x_2 + \cdots + x_t)^n = \sum \binom{n}{n_1n_2\ldots n_t} x_1^{n_1} x_2^{n_2} \cdots x_t^{n_t},
$
where the summation extends over all nonnegative integral solutions is \( n_1, n_2, \ldots, n_t \) of \( n_1 + n_2 + \cdots + n_t = n \).
\end{theorem}

\begin{theorem}
    \label{thm:5.5.1}
Let \( \alpha \) be a real number. Then, for all \( x \) and \( y \) with \( 0 \leq |x| < |y| \),
\[
(x + y)^\alpha = \sum_{k=0}^{\infty} \binom{\alpha}{k} x^k y^{\alpha-k},
\]
where
\[
\binom{\alpha}{k} = \frac{\alpha(\alpha - 1) \cdots (\alpha - k + 1)}{k!}.
\]
\end{theorem}

\begin{theorem}
    \label{thm:5.6.1}
Let \((X, \leq)\) be a finite partially ordered set, and let \( r \) be the largest size of a chain. Then \( X \) can be partitioned into \( r \) but no fewer antichains.
\end{theorem}


\begin{theorem}
    \label{thm:5.6.2}
Let \((X, \leq)\) be a finite partially ordered set, and let \(m\) be the largest size of an antichain. Then \(X\) can be partitioned into \(m\) but no fewer chains.
\end{theorem}



\chapter{The Inclusion-Exclusion Principle and Applications}

\begin{theorem}
    \label{thm:6.1.1}
The number of objects of the set $S$ that have none of the properties $P_1, P_2, \ldots, P_m$ is given by the alternating expression
\begin{align*}
    |\overline{A}_1 \cap \overline{A}_2 \cap \cdots \cap \overline{A}_m| 
    = & |S| - \sum |A_i| + \sum |A_i \cap A_j| - \sum |A_i \cap A_j \cap A_k| + \\ 
    & \cdots + (-1)^m |A_1 \cap A_2 \cap \cdots \cap A_m|, \tag{6.2}
\end{align*}
where the first sum is over all 1-subsets $\{i\}$ of $\{1,2,\ldots,m\}$, the second sum is over all 2-subsets $\{i,j\}$ of $\{1,2,\ldots,m\}$, the third sum is over all 3-subsets $\{i,j,k\}$ of $\{1,2,\ldots,m\}$, and so on until the $m$th sum over all $m$-subsets of $\{1,2,\ldots,m\}$ of which the only one is itself.
\end{theorem} 


\begin{corollary}
    \label{cor: 6.1.2}
The number of objects of $S$ which have at least one of the properties $P_1, P_2, \ldots, P_m$ is given by
\begin{align*}
    |A_1 \cup A_2 \cup \cdots \cup A_m| 
    = & \sum |A_i| - \sum |A_i \cap A_j| + \sum |A_i \cap A_j \cap A_k| - \\
    & \cdots + (-1)^{m+1} |A_1 \cap A_2 \cap \cdots \cap A_m|,
\tag{6.3}
\end{align*}

where the summations are as specified in Theorem~\ref{thm:6.1.1}.
\end{corollary}

\begin{theorem}
    \label{thm:6.3.1}
For $n \geq 1$,
\[
D_n = n! \left( 1 - \frac{1}{1!} + \frac{1}{2!} - \frac{1}{3!} + \cdots + (-1)^n \frac{1}{n!} \right).
\]
\end{theorem}

\begin{theorem}
    \label{thm:6.4.1}
The number of ways to place $n$ nonattacking, indistinguishable rooks on an $n$-by-$n$ board with forbidden positions equals
\[
n! - r_1 (n-1)! + r_2 (n-2)! - \cdots + (-1)^k r_k (n-k)! + \cdots + (-1)^n r_n.
\]
\end{theorem}

\begin{theorem}
    \label{thm:6.5.1}
For $n \geq 1$,
\begin{align*}
    Q_n = \quad & n! - \binom{n-1}{1}(n-1)! + \binom{n-1}{2}(n-2)! \\
    & - \binom{n-1}{3}(n-3)! + \cdots + (-1)^{n-1} \binom{n-1}{n-1}1!.
\end{align*}

\end{theorem}

\begin{theorem}
    \label{thm:6.6.1}
Let $(X, \leq)$ be a partially ordered set with a smallest element $0$. Let $\mu$ be its Möbius function, and let $F : X \to \Re$ be a real-valued function defined on $X$. Let the function $G : X \to \Re$ be defined by
\[
G(x) = \sum_{\{z : z \leq x\}} F(z), \quad (x \in X).
\]
Then
\[
F(x) = \sum_{\{y : y \leq x\}} G(y) \mu(y, x), \quad (x \in X).
\]
\end{theorem}

\begin{corollary}
    \label{cor: 6.6.2}
Let $X_n = \{1, 2, \ldots, n\}$ and let $F : \mathcal{P}(X_n) \to \Re$ be a function defined on the subsets of $X_n$. Let $G : \mathcal{P}(X_n) \to \Re$ be the function defined by
\[
G(K) = \sum_{L \subseteq K} F(L), \quad (K \subseteq X_n).
\]
Then
\[
F(K) = \sum_{L \subseteq K} (-1)^{|K| - |L|} G(L), \quad (K \subseteq X_n).
\]
\end{corollary}

\begin{theorem}
    \label{thm:6.6.3}
Let $(X, \leq_1)$ and $(Y, \leq_2)$ be two finite partially ordered sets with Möbius functions $\mu_1$ and $\mu_2$, respectively. Let $\mu$ be the Möbius function of the direct product of $(X, \leq_1)$ and $(Y, \leq_2)$. Then
\[
\mu\big((x, y), (x', y')\big) = \mu(x, x') \mu(y, y'), \quad \big((x, y), (x', y') \in X \times Y\big).
\tag{6.29}
\]
\end{theorem}

\begin{theorem}
    \label{thm:6.6.4}
Let $F$ be a real-valued function defined on the set of positive integers. Define a real-valued function $G$ on the positive integers by
\[
G(n) = \sum_{k : k \mid n} F(k).
\]
Then, for each positive integer $n$, we have
\[
F(n) = \sum_{k : k \mid n} \mu\left(n/k \right) G(k),
\]
where we write $\mu(n/k)$ for $\mu(1, n/k)$.
\end{theorem}

\chapter{Recurrence Relations and Generating Functions}
\begin{theorem}
 \label{thm:7.1.1}
The Fibonacci numbers satisfy the formula:
\[
f_n = \frac{1}{\sqrt{5}} \left( \frac{1 + \sqrt{5}}{2} \right)^n - \frac{1}{\sqrt{5}} \left( \frac{1 - \sqrt{5}}{2} \right)^n, \quad (n \geq 0).\tag{7.8}
\]
\end{theorem}


\begin{theorem}
 \label{thm:7.1.2}
The sums of the binomial coefficients along the diagonals of Pascal's triangle running upward from the left are Fibonacci numbers. More precisely, the $n$th Fibonacci number $f_n$ satisfies
\[
f_n = \binom{n-1}{0} + \binom{n-2}{1} + \binom{n-3}{2} + \cdots + \binom{n-t}{t-1},
\]
where $t = \left\lfloor \frac{n+1}{2} \right\rfloor$ is the floor of $\frac{n+1}{2}$.
\end{theorem}

\begin{theorem}
  \label{thm:7.2.1}
Let $n$ be a positive integer. Then
\[
g_n(x) = 1(1 + x)(1 + x + x^2)(1 + x + x^2 + x^3) \cdots (1 + x + x^2 + \cdots + x^{n-1})
\]
\[
= \frac{\prod_{j=1}^n (1 - x^j)}{(1 - x)^n}.\tag{7.14}
\]
\end{theorem}

\begin{theorem}
\label{thm:7.3.1}
 Let $S$ be the multiset $\{n_1 \cdot a_1, n_2 \cdot a_2, \dots, n_k \cdot a_k\}$, where $n_1, n_2, \dots, n_k$ are nonnegative integers. Let $h_n$ be the number of $n$-permutations of $S$. Then the exponential generating function $g^{(e)}(x)$ for the sequence $h_0, h_1, h_2, \dots, h_n, \dots$ is given by
\[
    g^{(e)}(x) = f_{n_1}(x) f_{n_2}(x) \cdots f_{n_k}(x), \tag{7.18}
\]
where, for $i = 1, 2, \dots, k$,
\[
    f_{n_i}(x) = 1 + x + \frac{x^2}{2!} + \cdots + \frac{x^n}{{n_i}!}. 
\]
\end{theorem}



\begin{theorem}
  \label{thm:7.4.1}
Let $q$ be a nonzero number. Then $h_n = q^n$ is a solution of the linear homogeneous recurrence relation
\[
    h_n - a_1 h_{n-1} - a_2 h_{n-2} - \cdots - a_k h_{n-k} = 0, \quad (a_k \neq 0, n \geq k) \tag{7.29}
\]
with constant coefficients if and only if $q$ is a root of the polynomial equation
\[
    x^k - a_1 x^{k-1} - a_2 x^{k-2} - \cdots - a_k = 0. \tag{7.30}
\]

If the polynomial equation has $k$ distinct roots $q_1, q_2, \dots, q_k$, then
\[
h_n = c_1 q_1^n + c_2 q_2^n + \cdots + c_k q_k^n. \tag{7.31}
\]
is the general solution of (7.29) in the following sense: No matter what initial values for 
$h_0, h_1, \dots, h_{k-1}$ are given, there are constants $c_1, c_2, \dots, c_k$ so that (7.31) 
is the unique sequence which satisfies both the recurrence relation (7.29) and the initial values.


\end{theorem}


\begin{theorem}
    \label{thm:7.4.2}
 Let $q_1, q_2, \dots, q_t$ be the distinct roots of the following characteristic equation of the linear homogeneous recurrence relation with constant coefficients:
\[
    h_n = a_1 h_{n-1} + a_2 h_{n-2} + \cdots + a_k h_{n-k}, \quad a_k \neq 0, \quad (n \geq k). \tag{7.39}
\]

If $q_i$ is an $s_i$-fold root of the characteristic equation of (7.39), the part of the general solution of this recurrence relation corresponding to $q_i$ is
\[
H_n^{(i)} = c_1 q_i^n + c_2 n q_i^n + \cdots + c_{s_i} n^{s_i-1} q_i^n,
\]
\[
= (c_1 + c_2 n + \cdots + c_{s_i} n^{s_i-1}) q_i^n.
\]

The general solution of the recurrence relation is
\[
h_n = H_n^{(1)} + H_n^{(2)} + \cdots + H_n^{(t)}.
\]

\end{theorem}



\begin{theorem}
    \label{thm:7.4.3}
Let 
\[
h_0, h_1, h_2, \dots, h_n, \dots
\]
be a sequence of numbers that satisfies the linear homogeneous recurrence relation
\[
    h_n + c_1 h_{n-1} + \cdots + c_k h_{n-k} = 0, \quad c_k \neq 0, \quad (n \geq k). \tag{7.42}
\]
of order $k$ with constant coefficients. Then its generating function $g(x)$ is of the form
\[
    g(x) = \frac{p(x)}{q(x)}, \tag{7.43}
\]
\end{theorem}



\begin{theorem}
\label{thm:7.6.1}
Let $h_n$ denote the number of ways of dividing a convex polygonal region with $n+1$ sides into triangular regions by inserting diagonals that do not intersect in the interior. Define $h_1 = 1$. Then $h_n$ satisfies the recurrence relation
\begin{align}
h_n &= h_1 h_{n-1} + h_2 h_{n-2} + \cdots + h_{n-1} h_1
\\
&= \sum_{k=1}^{n-1} h_k h_{n-k}, \quad (n \geq 2). \tag{7.54}
\end{align}
The solution of this recurrence relation is
\[
h_n = \frac{1}{n} \binom{2n-2}{n-1}, \quad (n = 1, 2, 3, \dots).
\]
\end{theorem}

\chapter{Special Counting Sequences}

\begin{theorem}
  \label{thm:8.1.1}
  The number of sequences 
  \begin{equation}\label{8.1}
  a_1, a_2, \ldots, a_{2n}
  \end{equation}
  of $2n$ terms that can be formed by using exactly $n$ +1s and exactly $n$ -1s whose partial sums 
  are always positive:
  \begin{equation}\label{8.2}
  a_1 + a_2 + \cdots + a_{k} \geq 0, \qquad  (k = 1, 2, \ldots, 2n)
  \end{equation}
  equals the nth Catalan number
  \begin{align*}
  C_n = \frac{1}{n+1} \binom{2n}{n}, \qquad (n \geq 0).
  \end{align*}
\end{theorem}

\begin{theorem}
  \label{thm:8.2.1}
  Let the geneml term of a sequence be a polynomial of degree $p$ in $n$:
  \[ h_n = a_pn^p + a_{p-1}n^{p-1} + \cdots + a_1n + a_0, \qquad (n \geq 0). \]
  Then $\Delta^{p+1}h_n = 0$ for all $n \geq 0$.
\end{theorem}

\begin{theorem}
  \label{thm:8.2.2}
  The geneml term of the sequence whose difference table has its Oth diagonal equal to
  \[ c_0, c_1, c_2, \ldots, c_p, 0, 0, 0, \ldots, \qquad \text{where} c_p \neq 0  \]
  is a polynomial in n of degree $p$ satisfying
  \begin{equation}\label{8.12}
    h_n = c_0\binom{n}{0} + c_1\binom{n}{1} + c_2\binom{n}{2} + \cdots + c_p\binom{n}{p}.
  \end{equation}
\end{theorem}

\begin{theorem}
  \label{thm:8.2.3}
  Assume that the sequence $h_0, h_1, h_2, \ldots, h_n, \ldots$ has a difference .table whose Oth 
  diagonal equals
  \[ c_0, c_1, c_2, \ldots, c_p, 0, 0, 0, \ldots. \]
  Then
  \begin{equation}
    \sum\limits_{k = 0}^{n}h_k = c_0\binom{n+1}{1} + c_1\binom{n+1}{2} + c_2\binom{n+1}{3} + \cdots 
    + c_p\binom{n+1}{p+1}.
  \end{equation}
\end{theorem}

\begin{theorem}
  \label{thm:8.2.4}
  If $1 \leq k \leq p- 1$, then
  \[ S(p, k) = kS(p-1, k) + S(p-1, k-1) \]
\end{theorem}

\begin{theorem}
  \label{thm:8.2.5}
  The Stirling number of the second kind $S(p, k)$ counts the number of partitions of a set of $p$ 
  elements into $k$ indistinguishable boxes in which no box is empty.
\end{theorem}

\begin{theorem}
  \label{thm:8.2.6}
  For each integer $k$ with $0 \leq k \leq p$, we have
  \[ S^{#}(p, k) = \sum\limits_{t = 0}^{k} (-1)^t \binom{k}{t}(k-t)^p; \]
  hence,
  \[ S(p, k) = \frac{1}{k!}\sum\limits_{t = 0}^{k} (-1)^t \binom{k}{t}(k-t)^ \]
\end{theorem}

\begin{theorem}
  \label{thm:8.2.7}
  If $p \geq 1$, then 
  \[ B_p = \binom{p-1}{0}B_0 + \binom{p-1}{1}B_1 + \cdots + \binom{p-1}{p-1}B_{p-1}. \]
\end{theorem}

\begin{theorem}
  \label{thm:8.2.8}
  If $1 \leq k \leq p- 1$, then
  \[ s(p, k) = (p-1)s(p-1, k) + s(p-1, k-1). \]
\end{theorem}

\begin{theorem}
  \label{thm:8.2.9}
  The Stirling number $s(p, k)$ of the first kind counts the number of arrangements of $p$ objects 
  into $k$ nonempty circular permutations.
\end{theorem}

\begin{theorem}
  \label{thm:8.3.2}
  Let $n$ be a positive integer. Let $p_n^s$ equal the number of self-conjugate partitions of $n$, 
  and let $p_n^t$ be the number of partitions of $n$ into distinct odd parts. Then
  \[ p_n^s = p_n^t. \]
\end{theorem}

\begin{theorem}
  \label{thm:8.3.3}
  Let $n$ be a positive integer. Let $p_n^{\textbf{o}}$ be the number of partitions of $n$ into odd
  parts, and let $p_n^{\textbf{d}}$ be the number of partitions of $n$ into distinct parts. Then
  \[ $p_n^{\textbf{o}}$ = $p_n^{\textbf{d}}$. \]
\end{theorem}

\begin{theorem}
  \label{thm:8.3.4}
  \[ \sum\limits_{n=0}^{\infty}p_nx^n = \prod\limits_{k=1}^{\infty}(1-x^k)^{-1} \]
\end{theorem}

\begin{theorem}
  \label{thm:8.3.5}
  Lexicographic order is a linear extension of the partial order of majorization on the set 
  $\mathcal{P}_n$ of partitions of a positive integer $n$.
\end{theorem}

\begin{theorem}
  \label{thm:8.3.6}
  Let $n$ be a positive integer. Let $p_n^{\prime}$ be the number of partitions of $n$ into an even 
  number of distinct parts, and let $p_n^{\prime\prime}$ be the number of partitions of $n$ into an 
  odd number of distinct parts. Then
  \[ p_n^{\prime} = p_n^{\prime\prime} \]
  where $e_n$ is an error term given by $e_n = (-1)^j$ if $n$ is of the form $j(3j\pm 1)/2$, and 0 
  otherwise.
\end{theorem}

\begin{theorem}
  \label{thm:8.5.1}
  The number of rectangular lattice paths from $(r, s)$ to $(p, q)$ equals the binomial coefficient
  \[ \binom{p-r+q-s}{p-r} = \binom{p-r+q-s}{q-s} \]
\end{theorem}

\begin{theorem}
  \label{thm:8.5.2}
  Let $n$ be a nonnegative integer. Then the number of subdiagonal rectangular lattice paths from 
  $(0,0)$ to $(n, n)$ equals the $n$th Catalan number
  \[ C_n = \frac{1}{n+1}\binom{2n}{n}. \]
\end{theorem}

\begin{theorem}
  \label{thm:8.5.3}
  Let $p$ and $q$ be positive integers with $p \geq q$. Then the number of subdiagonal rectangular 
  lattice paths from $(0,0)$ to $(p, q)$ equals
  \[ \frac{p-q+1}{p+1}\binom{p+q}{q}. \]
\end{theorem}

\begin{theorem}
  \label{thm:8.5.4}
  Let $r \leq \min\{p, q\}$. Then 
  \[ K(p, q : rD) = \begin{eqnarray}
    \begin{pmatrix}
       & p+q-r & \\
      p-r & q-r & r 
    \end{pmatrix}
  \end{eqnarray} = \frac{(p+q-r)!}{(p-r)!(q-r)!r!}, \]
  and 
  \[ K(p,q) = \sum\limits_{r=0}^{\min\{p, q\}}\frac{(p+q-r)!}{(p-r)!(q-r)!r!}. \]
\end{theorem}

\begin{theorem}
  \label{thm:8.5.5}
  Let $p$ and $q$ be positive integers with $p \geq q$, and let $r$ be a nonnegative integer with 
  $r \leq q$. Then
  \begin{align*}
    R(p, q : rD) &= \frac{p-q+1}{p-r+1}\frac{(p+q-r)!}{r!(p-r)!(q-r)!} \\
    &= \frac{p-q+1}{p-r+1}\begin{eqnarray}
      \begin{pmatrix}
         & p+q-r & \\
        r & (p-r) & (q-r)
      \end{pmatrix}
    \end{eqnarray}
  \end{align*}
  and
  \[R(p, q)=\sum\limits_{r=0}^q \frac{p-q+1}{p-r+1} \frac{(p+q-r)!}{r!(p-r)!(q-r)!}\]
\end{theorem}

\begin{theorem}
  \label{thm:8.5.6}
  The generating function for the sequence ($s_n: n \geq 1$) of small Schroder numbers is
  \[\sum\limits_{n=1}^{\infty} s_n x^n=\frac{1}{4}\left(1+x-\sqrt{x^2-6 x+1}\right). \]
\end{theorem}

\begin{theorem}
  \label{thm:8.5.7}
  The generating function for the sequence ($R_n: n \geq 0$) of large Schroder numbers is
  \[\sum\limits_{n=0}^{\infty} R_n x^n=\frac{1}{2x}\left(-(x - 1)-\sqrt{x^2-6 x+1}\right). \]
\end{theorem}

\begin{corollary}
  \label{cor:8.5.8}
  The large and small Schroder numbers are related by
  \[ R_n = 2s_{n+1}, \qquad (n \geq 1). \]
\end{corollary}

\begin{theorem}
  \label{thm:8.5.9}
  Let $n$ be a positive integer. Then the number of dissections of a convex polygonal region of 
  $n + 1$ sides equals the small Schroder number $s_n$.
\end{theorem}

\chapter{Systems of Distinct Representatives}

\begin{theorem}
  \label{thm:9.2.1}
  In order for the family $\mathcal{A} = (A_1, A_2 ,\ldots , A_n)$ of sets to have an $SDR$, it is 
  necessary that the following condition hold:\\
  (MC): For each $k=1,2, \ldots, n$ and each choice of $k$ distinct indices $i_1, i_2, \ldots, i_k$ 
  from $\{1,2, \ldots, n\}$,
  \begin{equation}\label{9.1}
    \left|A_{i_1} \cup A_{i_2} \cup \cdots \cup A_{i_k}\right| \geq k  
  \end{equation}
  in short, every $k$ sets of the family collectively contain at least $k$ elements.
\end{theorem}

\begin{theorem}
  \label{thm:9.2.2}
  The family $\mathcal{A} = (A_1, A_2 ,\ldots , A_n)$ of subsets of a set $Y$ has an SDR if and only
  if the marriage condition MC holds.
\end{theorem}

\begin{theorem}
  \label{thm:9.2.3}
  Let $\mathcal{A} = (A_1, A_2 ,\ldots , A_n)$ be a family of subsets of a finite set $Y$. Let $t$ 
  be an integer with $0 \leq t \leq n$. Then there exists a subfamily of $t$ sets of $\mathcal{A}$ 
  that has an SDR if and only if
  \begin{equation}\label{9.2}
    \left|A_{i_1} \cup A_{i_2} \cup \cdots \cup A_{i_k}\right| \geq k - (n -t) 
  \end{equation}
  for all $k$ with $k \geq n-t$ and all choices of $k$ distinct indices $i_1, i_2, \ldots, i_k$ 
  from $\{1,2, \ldots, n\}$.
\end{theorem}

\begin{theorem}
  \label{thm:9.2.4}
  Let $\mathcal{A} = (A_1, A_2 ,\ldots , A_n)$ be a family of subsets of a finite set $Y$. Then the 
  largest number of sets in a subfamily of $\mathcal{A}$ with an SDR equals the smallest value taken
  by the expression
  \begin{equation}\label{9.4}
    \left|A_{i_1} \cup A_{i_2} \cup \cdots \cup A_{i_k}\right| + n -k
  \end{equation}
  over all choices of $k = 1,2, \ldots ,n$ and all choices of $k$ indices $i_1, i_2, \ldots, i_k$ 
  with $1 \leq i_1 < i_2 <\ldots< i_k \leq n$.
\end{theorem}

\begin{theorem}
  \label{thm:9.3.1}
  For each preferential ranking matrix, there exists a stable complete marriage.
\end{theorem}

\begin{theorem}
  \label{thm:9.3.2}
  The stable complete marriage obtained from the deferred acceptance algorithm, with the women 
  choosing the men, is women-optimal. If the men choose the women in the deferred acceptance 
  algorithm, the resulting complete marriage is men-optimal.
\end{theorem}

\begin{corollary}
  \label{cor:9.3.3}
  In the women-optimal stable complete marriage, each man is paired with the woman he ranks lowest 
  among all the partners that are possible for him in a stable complete marriage.
\end{corollary}

\chapter{Combinatorial Designs}

\begin{theorem}
  \label{thm:10.1.1}
  The preceding algorithm terminates and computes the GCD of $a$ and $b$ correctly.
\end{theorem}

\begin{theorem}
  \label{thm:10.1.2}
  Let $n$ be an integer with $n \geq 2$ and let $a$ be a nonzero integer in $Z_n = \{O,1 ,\ldots, 
  n - 1\}$. Then $a$ has a multiplicative inverse in $Z_n$ if and only if the GCD of $a$ and $n$ is 
  $1$. If $a$ has a multiplicative inverse, then it is unique.
\end{theorem}

\begin{corollary}
  \label{cor:10.1.3}
  Let $n$ be a prime number. Then each nonzero integer in $Z_n$ has a multiplicative inverse.
\end{corollary}

\begin{theorem}
  \label{thm:10.2.1}
  In a BIBD, each variety is contained in
  \[ r = \frac{\lambda(v-1)}{k-1} \]
  blocks.
\end{theorem}

\begin{corollary}
  \label{cor:10.2.2}
  In a BIBD, we have
  \[ bk=vr. \]
\end{corollary}

\begin{corollary}
  \label{cor:10.2.3}
  In a BIBD, we have
  \[ \lambda < r. \]
\end{corollary}

\begin{theorem}
  \label{thm:10.2.4}
  In a BIBD, $b \geq v.$
\end{theorem}

\begin{theorem}
  \label{thm:10.2.5}
  Let $B$ be a subset of $k < v$ elements of $Z_v$ that forms a difference set mod $v$. Then the 
  blocks developed from $B$ as a starter block form an SBIBD with index
  \[ \lambda = \frac{k(k-1)}{v-1} \]
\end{theorem}

\begin{theorem}
  \label{thm:10.3.1}
  Let $\mathcal{B}$ be a Steiner triple system with pammeters $b, v, k = 3, r, \lambda$. Then
  \begin{equation}\label{10.4}
    r = \frac{\lambda(v-1)}{2}
  \end{equation}
  and 
  \begin{equation}\label{10.5}
    b = \frac{\lambda v(v-1)}{6}
  \end{equation}
  If the index is $\lambda = 1$, then there is a nonnegative integer $n$ such that $v = 6n +1$ or 
  $v = 6n +3$.
\end{theorem}

\begin{theorem}
  \label{thm:10.3.2}
  If there are Steiner triple systems of index $\lambda = 1$ with $v$ and $w$ varieties, 
  respectively, then there is a Steiner triple system of index $\lambda = 1$ with $vw$ varieties.
\end{theorem}

\begin{theorem}
  \label{thm:10.4.1}
  Let $n$ be a positive integer. Let $A$ be the $n$-by-$n$ array whose entry $a_{ij}$ in row $i$ 
  and column $j$ is
  \[ a_{i j}=i+j(\text{addition mod} \hspace{0.5em} n),(i, j=0,1, \ldots, n-1)\]
  Then $A$ is a Latin square of order $n$ based on $Z_n$.
\end{theorem}

\begin{theorem}
  \label{thm:10.4.2}
  Let $n$ be a positive integer and let $r$ be a nonzero integer in $Z_n$ such that the GCD of rand 
  $n$ is $1$. Let $A$ be the $n$-by-$n$ array whose entry $a_{ij}$ in row $i$ and column $j$ is
  \[ a_{i j}=r \times i+j(\text{addition mod} \hspace{0.5em} n),(i, j=0,1, \ldots, n-1) \]
  Then $A$ is a Latin square of order $n$ based on $Z_n$.
\end{theorem}

\begin{theorem}
  \label{thm:10.4.3}
  Let $n$ be a prime number. Then $L_n^1, L_n^2, \ldots , L_n^{n-1}$ are $n - 1$ MOLS of order $n$.
\end{theorem}

\begin{theorem}
  \label{thm:10.4.4}
  Let $n = p^k$ be an integer that is a power of a prime number $p$. Then there exist $n - 1$ MOLS 
  of order $n$. In fact, the $n - 1$ Latin squares (10.12) of order $n$ constructed from a finite 
  field with $n = p^k$ elements are $n - 1$ MOLS of order $n$.
\end{theorem}

\begin{theorem}
  \label{thm:10.4.5}
  Let $n \geq 2$ be an integer, and let $A_1, A_2,\ldots, A_k$ be $k$ MOLS of order $n$. Then $k 
  \leq n - 1$; that is, the largest number of MOLS of order $n$ is at most $n - 1$.
\end{theorem}

\begin{theorem}
  \label{thm:10.4.6}
  $N(n) \geq 2$ for each odd integer $n$.
\end{theorem}

\begin{theorem}
  \label{thm:10.4.7}
  If there is a pair of MOLS of order $m$ and there is a pair of MOLS of order $k$, then there is a 
  pair of MOLS of order $mk$. More generally,
  \[ N(mk) \geq \min\{N(m), N(k)\} \]
\end{theorem}

\begin{theorem}
  \label{thm:10.4.8}
  Let $n \geq 2$ be an integer and let
  $$
  n=p_1^{e_1} \times p_2^{e_2} \times \cdots \times p_k^{e_k}
  $$
  be the factorization of $n$ into distinct prime numbers $p_1, p_2, \ldots, p_k$. Then
  $$
  N(n) \geq \min \left\{p_i^{e_2}-1: i=1,2, \ldots, k\right\}
  $$
\end{theorem}

\begin{corollary}
  \label{cor:10.4.9}
  Let $n \geq 2$ be an integer that is not twice an odd number. Then there exists a pair of 
  orthogonal Latin squares of order $n$.
\end{corollary}

\begin{theorem}
  \label{thm:10.4.10}
  Let $n \geq 2$ be an integer. If there exist $n-1$ MOLS of order $n$, then there exists a 
  resolvable BIBD with parameters
  \begin{equation}\label{10.18}
    b=n^2+n, v=n^2, k=n^{\prime}, r=n+1, \lambda=1
  \end{equation}
  Conversely, if there exists a resolvable BIBD with parameters (10.18), then there exist $n-1$ 
  MOLS of order $n$.
\end{theorem}

\begin{theorem}
  \label{thm:10.4.11}
  Let $L$ be an $m$-by-$n$ Latin rectangle based on $Z_n$ with $m < n$. Then $L$ has a completion.
\end{theorem}

\begin{theorem}
  \label{thm:10.4.12}
  Let $L$ be a semi-Latin square of order $n$ and index $m$, where $m < n$. Then $L$ has a completion.
\end{theorem}

\chapter{Introduction to Graph Theory}

\begin{theorem}
  \label{thm:11.1.1}
  Let $G$ be a general graph. The sum
  \[ d_1 + d_2 + \cdots + d_n\]
  of the degrees of all the vertices of $G$ is an even number, and, consequently, the number of 
  vertices of $G$ with odd degree is even.
\end{theorem}

\begin{theorem}
  \label{thm:11.1.2}
  Two isomorphic general graphs have the same degree sequence, but two graphs with the same degree 
  sequence need not be isomorphic.
\end{theorem}

\begin{theorem}
  \label{thm:11.1.3}
  Let $G=(V, E)$ be a general graph. Then the vertex set $V$ can be uniquely partitioned into 
  nonempty parts $V_1, V_2, \ldots, V_k$ so that the following conditions are satisfied:
  (1) The general subgraphs $G_1=\left(V_1, E_1\right), G_2=\left(V_2, E_2\right), \ldots, G_k=
  \left(V_k, E_k\right)$ induced by $V_1, V_2, \ldots, V_k$, respectively, are connected.\\
  (2) For each $i \neq j$ and each pair of vertices $x$ in $V_i$ and $y$ in $V_j$, there is no walk 
  that joins $x$ and $y$.
\end{theorem}

\begin{theorem}
  \label{thm:11.1.4}
  Let $G$ and $G^{\prime}$ be two general graphs. Then the following are necessary conditions for 
  $G$ and $G^{\prime}$ to be isomorphic:
  (1) If $G$ is a graph, so is $G^{\prime}$.\\
  (2) If $G$ is connected, so is $G^{\prime}$. More generally, $G$ and $G^{\prime}$ have the same 
  number of connected components.\\
  (3) If $G$ has a cycle of length equal to some integer $k$, then so does $G^{\prime}$.\\
  (4) If $G$ has an (induced) general subgraph that is a complete graph $K_m$ of order $m$, so does
  $G^{\prime}$.
\end{theorem}

\begin{theorem}
  \label{thm:11.2.1}
  Let $G = (V, E)$ be a general graph and assume that the degree of each vertex is even. Then each 
  edge of $G$ belongs to a closed trail and hence to a cycle.
\end{theorem}

\begin{theorem}
  \label{thm:11.2.2}
  Let $G$ be a connected general gmph. Then $G$ has a closed Eulerian trail if and only if the 
  degree of each vertex is even.
\end{theorem}

\begin{theorem}
  \label{thm:11.2.3}
  Let $G$ be a connected general graph. Then $G$ has an open Eulerian trail if and only if there are
  exactly two vertices $u$ and $v$ of odd degree. Every open Eulerian trail in $G$ joins $u$ and $v$.
\end{theorem}

\begin{theorem}
  \label{thm:11.2.4}
  Let $G$ be a connected general graph and suppose that the number of vertices of $G$ with odd 
  degree is $m > O$. Then the edges of $G$ can be partitioned into $m/2$ open trails. It is 
  impossible to partition the edges of $G$ into fewer than $m/2$ open trails.
\end{theorem}

\begin{theorem}
  \label{thm:11.2.5}
  Let $G$ be a connected general graph having $K$ edges. Then there is a closed walk in $G$ of 
  length $2K$ in which the number of times an edge is used equals twice its multiplicity.
\end{theorem}

\begin{theorem}
  \label{thm:11.3.1}
  A connected graph of order $n \geq 3$ with a bridge does not have a Hamilton cycle.
\end{theorem}

\begin{theorem}
  \label{thm:11.3.2}
  Let $G$ be a graph of order $n \geq 3$ that satisfies the Ore property. Then $G$ has a Hamilton 
  cycle.
\end{theorem}

\begin{corollary}
  \label{cor:11.3.3}
  A graph of order $n \geq 3$, in which each vertex has degree at least $n/2$, has a Hamilton cycle.

\end{corollary}

\begin{theorem}
  \label{thm:11.3.4}
  A graph of order $n$, in which the sum of the degrees of each pair of nonadjacent vertices is at 
  least $n - 1$, has a Hamilton path.
\end{theorem}

\begin{theorem}
  \label{thm:11.4.1}
  A multigraph is bipartite if and only if each of its cycles has even length.
\end{theorem}

\begin{theorem}
  \label{thm:11.4.2}
  Let $G$ be a bipartite graph with bipartition $X, Y$. If $|X| \neq|Y|$, then $G$ does not have a 
  Hamilton cycle. If $|X|=|Y|$, then $G$ does not have a Hamilton path that begins at a vertex in 
  $X$ and ends at a vertex in $X$. If $X$ and $Y$ differ by at least 2 , then $G$ does not have a 
  Hamilton path. If $|X|=|Y|+1$, then $G$ does not have a Hamilton path that begins at $X$ and ends 
  at $Y$, or vice versa.
\end{theorem}

\begin{theorem}
  \label{thm:11.5.1}
  A connected graph of order $n$ has at least $n - 1$ edges. Moreover, for each positive integer $n$, 
  there exist connected graphs with exactly $n - 1$ edges. Removing any edge from a connected graph 
  of order $n$ with exactly $n-1$ edges leaves a disconnected graph, and hence each edge is a bridge.
\end{theorem}

\begin{theorem}
  \label{thm:11.5.2}
  A connected graph of order $n \geq 1$ is a tree if and only if it has exactly $n - 1$ edges.
\end{theorem}

\begin{theorem}
  \label{thm:11.5.3}
  Let $G$ be a connected graph and let $\alpha = \{x,y\}$ be an edge of $G$. Then $\alpha$ is a 
  bridge if and only if there does not exist a cycle of $G$ containing $\alpha$.
\end{theorem}

\begin{theorem}
  \label{thm:11.5.4}
  Let $G$ be a connected graph of order $n$. Then $G$ is a tree if and only if $G$ does not have 
  any cycles.
\end{theorem}

\begin{theorem}
  \label{thm:11.5.5}
  A graph $G$ is a tree if and only if every pair of distinct vertices $x$ and $y$ is joined by a 
  unique path. This path is necessarily a shortest path joining $x$ and $y$; that is, a path of 
  length $d(x, y)$.
\end{theorem}

\begin{theorem}
  \label{thm:11.5.6}
  Let $G$ be a tree of order $n \geq 2$. Then $G$ has at least two pendent vertices.
\end{theorem}

\begin{theorem}
  \label{thm:11.5.7}
  Every connected graph has a spanning tree.
\end{theorem}

\begin{theorem}
  \label{thm:11.5.8}
  Let $T$ be a spanning tree of a connected graph $G$. Let $\alpha = \{a,b\}$ be an edge of $G$ that
  is not an edge of $T$. Then there is an edge $\beta$ of $T$ such that the graph $T^{\prime}$ 
  obtained from $T$ by inserting $\alpha$ and deleting $\beta$ is also a spanning tree of $G$.
\end{theorem}

\begin{theorem}
  \label{thm:11.5.9}
  Let $T_1$ and $T_2$ be spanning trees of a connected graph $G$. Let $\beta$ be an edge of $T_1$. 
  Then there is an edge $\alpha$ of $T_2$ such that the graph obtained from $T_1$ by inserting 
  $\alpha$ and deleting $\beta$ is a spanning tree of $G$.
\end{theorem}

\begin{theorem}
  \label{thm:11.6.1}
  A neutral game is converted into a positive game if a new edge joining the distinguished vertices 
  $u$ and $v$ is added to the multigraph of the game.
\end{theorem}

\begin{theorem}
  \label{thm:11.6.2}
  The game determined by a multigraph $G = (V, E)$ with distinguished vertices $u$ and $v$ is a 
  positive game if and only if there is a subset $U$ containing $u$ and $v$ of the vertex set $V$ 
  such that the induced multisubgraph $G_U$ has two spanning trees, $T_1$ and $T_2$, with no common 
  edges.
\end{theorem}

\begin{theorem}
  \label{thm:11.7.1}
  Let $G = (V, E)$ be a graph. Then $G$ is connected if and only if the graph $T = (U, F)$ 
  constructed by carrying out the preceding algorithm is a spanning tree of $G$.
\end{theorem}

\begin{theorem}
  \label{thm:11.7.2}
  Let $G = (V, E)$ be a graph and let $u$ be any vertex of $G$. Then $G$ is connected if and only if
  the graph $T = (U, F)$ constructed by carrying out the BF-algorithm is a spanning tree of $G$. If 
  $G$ is connected, then, for each vertex $y$ of $G$, the distance in $G$ between $u$ and $y$ equals
  $D(y)$; and this is the same as the distance between $u$ and $y$ in $T$.
\end{theorem}

\begin{theorem}
  \label{thm:11.7.3}
  Let $G = (V, E)$ be a graph and let $u$ be any vertex of $G$. Then $G$ is connected if and only if
  the graph $T = (U, F)$, constructed by carrying out the preceding DF-algorithm, is a spanning tree
  of $G$.
\end{theorem}

\begin{theorem}
  \label{thm:11.7.4}
  Let $G = (V, E)$ be a weighted graph and let $u$ be any vertex of $G$. Then $G$ is connected if 
  and only if the graph $T = (U, F)$ obtained by carrying out the preceding algorithm is a spanning
  tree of $G$. If G is connected, then for each vertex $y$ of $G$, the weighted distance between $u$
  and $y$ equals $D(y)$, and this is the same as thf weighted distance between $u$ and $y$ in the 
  weighted tree $T$.
\end{theorem}

\begin{theorem}
  \label{thm:11.7.5}
  Let $G = (V, E)$ be a weighted connected graph with weight function $c$. Then the preceding greedy
  algorithm constructs a minimum weight spanning tree $T = (V, F)$ of $G$.
\end{theorem}

\begin{theorem}
  \label{thm:11.7.6}
  Let $G = (V, E)$ be a weighted graph with weight function $c$. Then Prim's algorithm constructs a 
  minimum weight spanning tree $T = (V, F)$ of $G$.
\end{theorem}


\chapter{More on Graph Theory}

\begin{theorem}
  \label{thm:12.1.1}
  Let $G$ be a graph of order $n \geq 1$. Then
  $$
  1 \leq \chi(G) \leq n
  $$
  Moreover, $\chi(G)=n$ if and only if $G$ is a complete graph, and $\chi(G)=1$ if and only if $G$ 
  is a null graph.
\end{theorem}

\begin{corollary}
  \label{cor:12.1.2}
  Let $G$ be a graph and let $H$ be a subgraph of $G$. Then $\chi(G) \geq \chi(H)$. If $G$ has a 
  subgraph equal to a complete graph $K_p$ of order $p$, then
  $$
  \chi(G) \geq p
  $$
\end{corollary}

\begin{corollary}
  \label{cor:12.1.3}
  Let $G=(V, E)$ be a graph of order $n$ and let $q$ be the largest order of an induced subgraph of 
  $G$ equal to a null graph $N_q$. Then
  $$
  \chi(G) \geq\left\lceil\frac{n}{q}\right\rceil .
  $$
\end{corollary}

\begin{theorem}
  \label{thm:12.1.4}
  Let $G$ be a graph with at least one edge. Then $\chi(G) = 2$ if and only if $G$ is bipartite.
\end{theorem}

\begin{theorem}
  \label{thm:12.1.5}
  Let $G$ be a graph for which the maximum degree of a vertex is $\Delta$. Then the greedy algorithm
  produces a $(\Delta+1)$-coloring of the vertices of $G$, and hence
  $$
  \chi(G) \leq \Delta+1
  $$
\end{theorem}

\begin{theorem}
  \label{thm:12.1.6}
  Let $G$ be a connected graph for which the maximum degree of a vertex is $\Delta$. If $G$ is 
  neither a complete graph $K_n$ nor an odd cycle graph $C_n$, then $\chi(G) \leq \Delta$.
\end{theorem}

\begin{theorem}
  \label{thm:12.1.7}
  Let $T$ be a tree of order $n$. Then
  \[ p_T(k) = k(k-1)^{n-1}. \]
\end{theorem}

\begin{theorem}
  \label{thm:12.1.8}
  Let $G$ be a graph of order $n \geq 1$. Then the number of $k$-colorings of $G$ is a polynomial 
  in $k$ of degree equal to $n$ (with leading coefficient equal to $1$) and this polynomial-the 
  chromatic polynomial of $G$-is computed correctly by the preceding algorithm. 
\end{theorem}

\begin{theorem}
  \label{thm:12.1.9}
  Let $G$ be a graph and assume that $G$ contains a subgraph $H$ equal to a complete graph $K_r$. 
  Then the chromatic polynomial of $G$ is divisible by the chromatic polynomial $[k]_r$ of $K_r$.
\end{theorem}

\begin{theorem}
  \label{thm:12.1.10}
  Let $U$ be an articulation set of $G$ and suppose that the induced subgraph $G_U$ is a complete graph $K_r$. Let the connected components of $G_{V-U}$ be the induced subgraphs $G_{U_1}, \ldots, G_{U_t}$. For $i=1, \ldots, t$, let $H_i=G_{U \cup U_i}$ be the subgraph of $G$ induced by $U \cup U_i$. Then
  $$
  p_G(k)=\frac{p_{H_1}(k) \times \cdots \times p_{H_t}(k)}{\left([k]_r\right)^{t-1}}
  $$
  In particular, the chromatic number of $G$ is the largest of the chromatic numbers of $H_1, \ldots, H_t$.
\end{theorem}

\begin{theorem}
  \label{thm:12.2.1}
  Let $G$ be a plane-graph of order $n$ with $e$ edge-curves and assume that $G$ is connected. Then the number $r$ of regions into which $G$ divides the planr satisfies
  \begin{equation}\label{12.4}
    r=e-n+2
  \end{equation}
\end{theorem}

\begin{theorem}
  \label{thm:12.2.2}
  Let $G$ be a connected planar graph. Then there is a vertex of $G$ whose degree is at most $5$.
\end{theorem}

\begin{theorem}
  \label{thm:12.2.3}
  A graph $G$ is planar if and only if it does not have a subgraph that is a subdivision of a $K_5$
  or of a $K_{3,3}$.
\end{theorem}

\begin{theorem}
  \label{thm:12.2.4}
  A graph $G$ is planar if and only if it does not contain a subgraph that contracts to a $K_5$ or
  a $K_{3,3}$.
\end{theorem}

\begin{theorem}
  \label{thm:12.3.1}
  Let there be given a $k$-coloring of the vertices of a graph $H=(U, F)$. Let two of the colors be red and blue, and let $W$ be the subset of vertices in $U$ that are assigned either the color red or the color blue. Let $H_{r, b}$ be the subgraph of $H$ induced by the vertices in $W$ and let $C_{r, b}$ be a connected component of $H_{r, b}$. Interchanging the colors red and blue assigned to the vertices of $C_{r, b}$, we obtain another $k$-coloring of $H$.
\end{theorem}

\begin{theorem}
  \label{thm:12.3.2}
  The chromatic number of a planar graph is at most $5$.
\end{theorem}

\begin{theorem}
  \label{thm:12.3.3}
  Hadwiger's conjecture holds for $p = 5$ if and only if every planar graph has a $4$-coloring.
\end{theorem}

\begin{theorem}
  \label{thm:12.3.4}
  Let $p \leq 3$. If $G$ is a connected graph with chromatic number $\chi(G) \geq p$, then $G$ can be contracted to a $K_p$.
\end{theorem}

\begin{theorem}
  \label{thm:12.4.1}
  Let $G$ be a connected graph of order $n \geq 2$. Then
  $$\text{dom}(G) \leq \left\lfloor \frac{n}{2} \right\rfloor.$$
\end{theorem}

\begin{theorem}
  \label{thm:12.4.2}
  A graph $G$ is $\chi$-perfect if and only if it is $\theta$-perfect. Equivalently, $G$ is $\chi$-perfect if and only if $\bar{G}$ is $\chi$-perfect.
\end{theorem}

\begin{theorem}
  \label{thm:12.4.3}
  Every interval graph is a chordal graph.
\end{theorem}

\begin{theorem}
  \label{thm:12.4.4}
  Let $G=(V, E)$ be a connected graph and let $U$ be an articulation set of $G$ such that the subgraph $G_U$ induced by $U$ is a complete graph. Let the connected components of the induced subgraph $G_{V-U}$ be $G_1=\left(U_1, E_1\right), \ldots, G_t=\left(U_t, E_t\right)$. Assume that the induced graphs $G_{U_2 \cup U}$ satisfy
  $$
  \chi\left(G_{U_2 \cup U}\right)=\omega\left(G_{U_2 \cup U}\right) \quad(i=1,2, \ldots, t)
  $$
  Then
  $$
  \chi(G)=\omega(G)
  $$
\end{theorem}

\begin{theorem}
  \label{thm:12.4.5}
  Let $G = (V, E)$ be a connected chordal graph and let $U$ be a minimal articulation set of $G$. Then the subgraph $G_U$ induced by $U$ is a complete graph.
\end{theorem}

\begin{theorem}
  \label{thm:12.4.6}
  Every chordal graph is perfect.
\end{theorem}

\begin{corollary}
  \label{cor:12.4.7}
  Every interval graph is a perfect graph.
\end{corollary}

\begin{theorem}
  \label{thm:12.5.1}
  Let $G=(V, E)$ be a bipartite graph with bipartition $X, Y$ with as sociated family $\mathcal{A}_G$ of subsets of $Y$. Let $t$ be a positive integer. Then from a subfamily
  \begin{equation}\label{12.10}
    \left(A_{i_1}, A_{i_2}, \ldots, A_{i_t}\right) \text { of } t \text { sets of } \mathcal{A}_G \text { with an } S D R\left(e_{i_1}, e_{i_2}, \ldots, e_{i_t}\right)
  \end{equation}
  we obtain a matching
  \begin{equation}\label{12.11}
    \left\{x_{i_1}, e_{i_1}\right\},\left\{x_{i_2}, e_{i_2}\right\}, \ldots,\left\{x_{i_t}, e_{i_t}\right\} \text { of } G \text { of } t \text { edges. }
  \end{equation}
  Conversely, from a matching (12.11) of $G$ of $t$ edges, we get a subfamily (12.10) of $\mathcal{A}_G$ of $t$ sets with $\left(e_{i_1}, e_{i_2}, \ldots, e_{i_t}\right)$ as $S D R$.
  Thus the largest number of sets in a subfamily of $\mathcal{A}_G$ with an $S D R$ equals thr matching number $\rho(G)$ of $G$.
\end{theorem}

\begin{theorem}
  \label{thm:12.5.2}
  Let $G = (V, E)$ be a graph. Then a subset $W$ of the set $V$ of vertices is a cover if and only if the complementary set of vertices $V \setminus W$ is an independent set.
\end{theorem}

\begin{theorem}
  \label{thm:12.5.3}
  Let $G=(V, E)$ be a bipartite graph. Then
  \begin{equation}\label{12.13}
    \rho(G)=c(G)
  \end{equation}
  that is, the largest number of edges in a matching equals the smallest number of vertices in a cover.
\end{theorem}

\begin{theorem}
  \label{thm:12.5.4}
  Let $G=(V, E)$ be a graph. Then $G$ has a perfect matching if and only if
  \begin{equation}\label{12.15}
    \text { oc }\left(G_{V \backslash U}\right) \leq|U| \text { for every } U \subseteq V
  \end{equation}
  that is, removing a set of vertices does not create more odd components than the number of vertices removed.
\end{theorem}

\begin{theorem}
  \label{thm:12.5.5}
  Let $G(V, E)$ be a graph with $n$ vertices. Then
  $$
  \rho(G)=\min \left\{n-\left(o c\left(G_{V \backslash U}\right)-|U|\right)\right\}
  $$
  where the minimum is taken over all $U \subseteq V$.
\end{theorem}

\begin{theorem}
  \label{thm:12.6.1}
  Let $G$ be a graph of order $n$. Then
  $$
  0 \leq \kappa(G) \leq n-1
  $$
  with equality on the left if and only if $G$ is disconnected and with equality on the right if and only if $G$ is a complete graph.
\end{theorem}

\begin{theorem}
  \label{thm:12.6.2}
  For each graph $G$, we have
  $$ \kappa(G) \leq \lambda(G) \leq \delta(G) $$
\end{theorem}

\begin{theorem}
  \label{thm:12.6.3}
  Let $G$ be a graph of order $n \geq 3$. Then the following three assertions are equivalent:
  (1) $G$ is 2-connected.
  (2) $G$ is connected and does not have an articulation vertex.
  (3) For each triple of vertices $a, b, c$, there is a path joining a and b that does not contain $c$.
\end{theorem}

\begin{theorem}
  \label{thm:12.6.4}
  Let $G=(V, E)$ be a connected graph of order $n \geq 2$, and let
  $$
  G_{U_1}=\left(U_1, E_1\right), G_{U_2}=\left(U_2, E_2\right), \ldots, G_{U_r}=\left(U_r, E_r\right)
  $$
  be the blocks of $G$. Then $E_1, E_2, \ldots, E_r$ is a partition of the set $E$ of edges of $G$, and each pair of blocks has at most one vertex in common.
\end{theorem}

\begin{theorem}
  \label{thm:12.6.5}
  Let $G = (V, E)$ be a graph of order $n \geq 3$. Then $G$ is $2$-connected if and only if, for each pair $a, b$ of distinct vertices, there is a cycle containing both $a$ and $b$.
\end{theorem}

\begin{corollary}
  \label{cor:12.6.6}
  Let $G$ be a graph with at least three vertices. Then $G$ is $2$-connected if and only if, for each pair $a, b$ of distinct vertices, there are two paths joining $a$ and $b$ whose only common vertices are $a$ and $b$.
\end{corollary}

\begin{theorem}
  \label{thm:12.6.7}
  Let $k$ be a positive integer and let $G$ be a graph of order $n \geq k +1$. Then $G$ is $k$-connected if and only if, for each pair $a, b$ of distinct vertices, there are $k$ paths joining $a$ and $b$ such that each pair of paths has only the vertices $a$ and $b$ in common.
\end{theorem}

\chapter{Digraphs and Networks}

\begin{theorem}
  \label{thm:13.1.1}
  In a general digraph the sum of the indegrees of the vertices equals the sum of the outdegrees, and each is equal to the number of arcs.
\end{theorem}

\begin{theorem}
  \label{thm:13.1.2}
  Let $D$ be a connected digraph. Then $D$ has a closed Eulerian directed trail if and only if the indegree of each vertex equals the outdegree.
\end{theorem}

\begin{theorem}
  \label{thm:13.1.3}
  Let $D$ be a connected digraph and let $x$ and $y$ be distinct vertices of $D$. Then there is a directed Eulerian trail from $x$ to $y$ if and only if
  (i) the outdegree of $x$ exceeds its indegree by 1 ;
  (ii) the indegree of $y$ exceeds its outdegree by 1 ;
  (iii) for each vertex $z \neq x, y$, the indegree of $z$ equals its outdegree.
\end{theorem}

\begin{theorem}
  \label{thm:13.1.4}
  Let $D$ be a strongly connected digraph without any loops. If, for each vertex $x$, we have
  $$
  (\text { outdegree of } x)+(\text { indegree of } x) \geq n,
  $$
  then $D$ has a directed Hamilton cycle.
\end{theorem}

\begin{theorem}
  \label{thm:13.1.5}
  Every tournament has a Hamilton path.
\end{theorem}

\begin{theorem}
  \label{thm:13.1.6}
  Let $G = (V, E)$ be a connected graph. Then $G$ has a strongly connected orientation if and only if $G$ does not have any bridges.
\end{theorem}

\begin{lemma}
  \label{lem:13.1.7}
  Let $D$ be a digraph in which each vertex has outdegree at least $l$. Then there is a directed cycle in $D$.
\end{lemma}

\begin{corollary}
  \label{cor:13.1.8}
  Let $X$ be a set of $n$ elements and let $f: X \rightarrow X$ be a one-to-one function. Let $D_f=\left(X, A_f\right)$ be the digraph whose set of arcs is
  $$
  A_f=\{(x, f(x)): x \text { in } X\}
  $$
  Then the arcs of $D_f$ can be partitioned into directed cycles with each vertex belonging to exactly one directed cycle.
\end{corollary}

\begin{theorem}
  \label{thm:13.1.9}
  Every trading problem has a core allocation.
\end{theorem}

\begin{lemma}
  \label{lem:13.2.1}
  Let $f$ be a flow in the network $N=(V, A, s, t, c)$ and let $U$ be a set of vertices containing the source $s$ but not the target $t$. Then
  $$
  \sum\limits_{\alpha \in \vec{U}} f(\alpha)-\sum\limits_{\alpha \in \overleftarrow{U}} f(\alpha)=\sum\limits_{\iota(\alpha)=s} f(\alpha)-\sum\limits_{\tau(\alpha)=s} f(\alpha)
  $$
\end{lemma}

\begin{lemma}
  \label{lem:13.2.2}
  Let $N=(V, A, s, t, c)$ be a network with $C$ a minimal cut. Let U be the set of all vertices $x$ for which there exists a path from the source $s$ to $x$ thal contains no arc in $C$. Then $\vec{U}$ is a cut and $C=\vec{U}$.
\end{lemma}

\begin{theorem}
  \label{thm:13.2.3}
  Let $N=(V, A, s, t, c)$ be a network. Then the maximum value of a flow in $N$ equals the minimum capacity of a cut in $N$. In other words, the value of a maximum flow equals the capacity of a minimum cut. If the capacities of all the arcs are integers, then there is a maximum flow all of whose values are integers as well.
\end{theorem}

\begin{theorem}
  \label{thm:13.2.4}
  Let $s$ and $t$ be distinct vertices of a digraph $D = (V, A)$. Then the maximum number of pairwise arc-disjoint paths from $s$ to $t$ equals the minimum number of arcs in an st-separating set.
\end{theorem}

\begin{theorem}
  \label{thm:13.2.5}
  Let $G$ be a bipartite graph. Then $\rho(G) =c(G)$.
\end{theorem}

\begin{theorem}
  \label{thm:13.3.1}
  Let $M$ be a matching in the bipartite graph $G$. Then $M$ is a max-matching if and only if there does not exist an $M$-alternating path.
\end{theorem}

\begin{theorem}
  \label{thm:13.3.2}
  Assume that nonbreakthrough has occured in the matching algorithm. Let $X^{u n}$ consist of all the unlabeled vertices of $X$, let $Y^{\text {lab }}$ consist of all the labeled vertices of $Y$, and let $S=X^{u n} \cup Y^{l a b}$. Then both of the following hold:
  (i) $S$ is a min-cover of the bipartite graph $G$;
  (ii) $|M|=|S|$ and $M$ is a max-matching.
\end{theorem}

\chapter{Po\'lya Counting}

\begin{theorem}
  \label{thm:14.2.1}
  For each coloring $\mathbf{c}$, the stabilizer $G(\mathbf{c})$ of $\mathbf{c}$ is a permutation group. Moreover, for any permutations $f$ and $g$ in $G, g * \mathbf{c}=f * \mathbf{c}$ if and only if $f^{-1} \circ g$ is in $G(\mathbf{c})$.
\end{theorem}

\begin{corollary}
  \label{cor:14.2.2}
  Let $\mathbf{c}$ be a coloring in $\mathcal{C}$. The number
  $$
  \mid\{f * \mathbf{c}: f \text { in } G\} \mid
  $$
  of different colorings that are equivalent to $\mathbf{c}$ equals the number
  $$
  \frac{|G|}{|G(\mathbf{c})|}
  $$
  obtained by dividing the number of permutations in $G$ by the number of permutations in the stabilizer of $\mathbf{c}$.
\end{corollary}

\begin{theorem}
  \label{thm:14.2.3}
  Let $G$ be a group of permutations of $X$ and let $\mathcal{C}$ be a set of colorings of $X$ such that $f * \mathbf{c}$ is in $\mathcal{C}$ for all $f$ in $G$ and all $\mathbf{c}$ in $\mathcal{C}$. Then the number $N(G, \mathcal{C})$ of nonequivalent colorings in $\mathcal{C}$ is given by
  \begin{equation}\label{14.7}
    N(G, \mathcal{C})=\frac{1}{|G|} \sum_{f \in G}|\mathcal{C}(f)|
  \end{equation}
  In words, the number of nonequivalent colorings in $\mathcal{C}$ equals the average of the number of colorings fixed by the permutations in $G$.   
\end{theorem}

\begin{theorem}
  \label{thm:14.3.1}
  Let $f$ be a permutation of a set $X$. Suppose we have $k$ colors available with which to color the elements of $X$. Let $\mathcal{C}$ be the set of all colorings of $X$. Then the number of colorings that are fixed by $f$ satisfies
  $$
  |\mathcal{C}(f)|=k^{\#(f)}
  $$
\end{theorem}

\begin{theorem}
  \label{thm:14.3.2}
  Let $X$ be a set of $n$ elements, and suppose we have a set of $k$ colors available with which to color the elements of $X$. Let $\mathcal{C}$ be the set of all $k^n$ colorings of $X$. Let $G$ be a group of permutations of $X$. Then the number of nonequivalent colorings is the number
  $$
  N(G, \mathcal{C})=P_G(k, k, \ldots, k)
  $$
  obtained by substituting $z_i=k,(i=1,2, \ldots, n)$ into the cycle index of $G$.
\end{theorem}

\begin{theorem}
  \label{thm:14.3.3}
  Let $X$ be a set of elements and let $G$ be a group of permutations of $X$. Let $\left\{u_1, u_2, \ldots, u_k\right\}$ be a set of $k$ colors, and let $\mathcal{C}$ be a set of all colorings of $X$. Then the generating function for the number of nonequivalent colorings of $\mathcal{C}$ according to the number of colors of each kind is the expression
  \begin{equation}\label{14.20}
    P_G\left(u_1+\cdots+u_k, u_1^2+\cdots+u_k^2, \ldots, u_1^n+\cdots+u_k^n\right)
  \end{equation}
  obtained from the cycle index $P_G\left(z_1, z_2, \ldots, z_n\right)$ by making the substitutions
  $$
  z_j=u_1^j+\cdots+u_k^j \quad(j=1,2, \ldots, n)
  $$

  In other words, the coefficient of
  $$
  u_1^{p_1} u_2^{p_2} \cdots u_k^{p_k}
  $$
  in (14.20) equals the number of nonequivalent colorings in $\mathcal{C}$ with $p_1$ elements of $X$ colored $u_1, p_2$ elements colored $u_2, \ldots, p_k$ elements colored $u_k$.
\end{theorem}


















